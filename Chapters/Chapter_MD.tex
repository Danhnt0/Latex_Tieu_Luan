
\chapter*{Mở đầu}


\addcontentsline{toc}{chapter}{Mở đầu}


Alan Turing, một nhà toán học lỗi lạc, người đã phá vỡ cỗ máy mã hóa Enigma của Đức quốc xã, đã đưa ra một câu hỏi làm thay đổi lịch sử, “Máy móc có thể suy nghĩ không?” vào năm 1950. Nghiên cứu thực sự bắt đầu vào năm 1956, 
tại một hội nghị được tổ chức tại Đại học Dartmouth (rất nhiều phát minh đã được đưa ra, nhờ Ivy League). Một vài người tham dự hội nghị là những người đã đưa ra ý tưởng và cũng là cái tên “Trí tuệ nhân tạo”. Nhưng vì toàn bộ 
ý tưởng là mới, mọi người đã không mua ý tưởng đó và tài trợ cho nghiên cứu sâu hơn đã bị cắt giảm. Giai đoạn này, những năm 1950 – 1980 được gọi là “AI Winter”. Tuy nhiên, vào đầu những năm 1980, chính phủ Nhật Bản đã nhìn thấy 
tương lai của AI và bắt đầu tài trợ trở lại cho lĩnh vực này. Vì điều này được kết nối với các lĩnh vực điện tử và khoa học máy tính, nên cũng có sự gia tăng đột biến trong các lĩnh vực đó. Máy AI đầu tiên được giới thiệu ra thế 
giới vào năm 1997; Deep Blue của IBM đã trở thành máy tính đầu tiên đánh bại một nhà vô địch cờ vua khi đánh bại đại kiện tướng người Nga Garry Kasparov. Ngày nay AI đóng góp hầu hết vào các lĩnh vực trong đời sống như y tế, giáo 
dục, an ninh, kinh tế và kể cả quân sự. Qua đó cho thấy được tầm quan trọng của AI.

Trong quá trình hình thành và phát triển từ những năm 50 của thế kỷ trước tới nay, có rất nhiều thư viện hay API được nghiên cứu và xây dựng để hỗ trợ cho việc phát triển AI. Trong đó có rất nhiều thư viện
khá phổ biến và mạnh mẽ như pytorch, pandas, SciPy, Keras, Theano... đặc biệt không thể thiếu thư viện TensorFlow. TensorFlow là một thư viện mã nguồn mở được phát triển bởi Google. Nó được thiết kế để
hỗ trợ việc phát triển các mô hình Machine Learning và Deep Learning. TensorFlow cũng có thể được sử dụng để phát triển các mô hình Deep Learning cho các thiết bị di động như Android và iOS. 

Một lĩnh vực khá thú vị trong AI cũng như học máy(ML) là thị giác máy tính, nghĩa là chúng ta sẽ làm cho máy tính có thể nhận biết, quan sát, và hiểu được các nội dung của các bức ảnh hoặc video. Để làm được điều đó,
chúng ta cần phải huấn luyện một mô hình học máy để nó có thể nhận biết được các đặc trưng của các bức ảnh. Trong tiểu luận này chúng ta sẽ tìm hiểu về một bài toán nhận biết chữ số viết tay với TensorFlow. Bài toán này sẽ được 
giải quyết bằng cách sử dụng thuật toán Neural Struct Learning và một mô hình mạng neural. Bài toán sử dụng bộ dữ liệu có sẵn của TensorFlow MNIST, là một bộ dữ liệu gồm các chữ số viết tay khác nhau đề huấn luyện mô hình.

Cấu trúc của bài tiểu luận này gồm ba chương:

\begin{itemize}
\item Chương 1: Tổng quan

     Ở chương này, chúng ta sẽ đi qua tổng qua về trí tuệ nhân tạo, học máy, học sâu và giới thiệu về thư viện TensorFlow.
     Ngoài ra chúng ta cũng sẽ tìm hiểu về thuật toán tối ưu Adam, thuật toán sẽ được sử dụng để huấn luyện mô hình học máy trong bài tiểu luận này.
\item Chương 2: Neural Structured Learning With Python/TensorFlow

    Ở chương này, chúng ta sẽ tìm hiểu về thuật toán Neural Structured Learning. Và khái quát bài toán cũng như sử dụng Adversarial Learning để giải quyết bài toán đề ra.

\item Chương 3: Nhận diện chữ số viết tay

    Ở chương này, chúng ta sẽ triển khai các mô hình và đào tạo nó, một mô hình cơ bản và một mô hình nâng cao sử dụng thuật toán Adversarial Learning để xem xét được ưu điểm 
    và sự khác biệt của mô hình Adversarial Learning. Sau đó chúng ta sẽ dùng hai mô hình vừa huấn luyện này để nhận biết một số các chữ số viết tay từ ảnh chụp trực tiếp.
\end{itemize}


