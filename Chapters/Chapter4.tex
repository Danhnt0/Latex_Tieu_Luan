
\chapter{Kết luận} % Tên của chương

\label{Chapter3} % Thay X bằng số chương tương ứng; để trích dẫn chương này ở chỗ nào đó trong bài, hãy sử dụng lệnh \ref{ChapterX} 

Qua các chương trên chúng ta đã có cái nhìn tổng quan về TensorFlow, Neural Structured Learning. Chúng ta đã cài đặt và sử dụng  Neural Structured Learning
để xử lý bài toán phân loại chữ số viết tay. Chúng ta biết rằng Neural Structured Learning được khái quát hóa bằng hai phần chính là Neural Graph Learning và
Adversarial Learning tuy nhiên ở chương 2 chúng ta mới chỉ sử dụng Adversarial Learning để thử nghiệm, quan sát và đánh giá nó. Và ở mục 2.2 chúng ta đã minh chứng 
rằng Neural Structured Learning có thể giúp cải thiện độ chính xác của mô hình và giúp các mô hình hoạt động tốt hơn trên những tập dữ liệu kém.

Ở mục 1.1.2 chúng ta thấy rằng chỉ cần thêm một nhiễu rất nhỏ mà mắt thường không thể phát hiện được, ngay lập tức đã đánh lừa được mô hình cơ bản phán đoán sai.
Điều này mang lại nhiều nguy cơ cho các hệ thống sử dụng mạng neural để phát hiện, phân loại hình ảnh như camera an ninh hay xe tự lái vì chỉ cần các bức ảnh mà camera
thu thập được không đủ tốt hoặc có nhiễu thì sẽ gây ra sai lầm cho các phán đoán cho hệ thống. Đối với các ứng dụng như nhận dạng hay phân loại đồ vật thì các sai sót xảy ra ít gây nghiêm trọng hơn, nhưng những ứng dụng như xe tự
lái, thu thập và phân tích hình ảnh theo thời gian thực, một phán đoán sai như không phát hiện người đi đường thì có thể gây ra hậu quả lớn. 
Vì vậy việc tìm ra các cách để cải thiện độ chính xác của mô hình trên các tập dữ liệu kém tương tự như Adversarial Learning là rất cần thiết.

\url{https://github.com/Danhnt0/Latex_Tieu_Luan}