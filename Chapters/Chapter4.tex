
\chapter*{Kết luận} % Tên của chương
\addcontentsline{toc}{chapter}{Kết luận}

\label{Chapter4} % Thay X bằng số chương tương ứng; để trích dẫn chương này ở chỗ nào đó trong bài, hãy sử dụng lệnh \ref{ChapterX} 

Trong bài tiểu luận này, các mô hình Neural Structured Learning sử dụng thư viện TensorFlow và Python đã được nghiên cứ và triển khai với bài toán phân loại chữ số viết tay.
Cụ thể là mô hình phân loại hình ảnh sử dụng thuật toán Adversarial Learning được huấn luyện với bộ dữ liệu MNIST có sẵn của TensorFlow. Sau đó, mô hình vừa được huấn
luyện được sử dụng để nhận diện các chữ số viết tay trong các bức ảnh được chụp trực tiếp. Từ kết quả thu được trên tập dữ liệu test có sẵn và dữ liệu ảnh chụp trực tiếp, chúng ta đã thấy rằng
rằng Neural Structured Learning có thể giúp cải thiện độ chính xác của mô hình và giúp các mô hình hoạt động tốt hơn trên những tập dữ liệu kém như những gì chúng ta đã thảo luận ở chương 2.

Các kết quả cụ thể là: độ chính xác của mô hình cơ bản bị giảm từ 99,15\% trên tập dữ liệu chuẩn xuống còn 57,7\% trên tập dữ liệu "nhiễu loạn". Trong khi đó độ chính xác của mô hình Adversarial Learning
giảm từ 99,31\% trên tập dữ liệu chuẩn xuống 96,2\% trên tập dữ liệu "nhiễu loạn". 
Đối với các chữ số viết tay được chụp trực tiếp thì mô hình Adversarial Learning cũng cho thấy sự cải thiện của mình so với mô hình cơ bản khi chỉ nhận diện sai 1 so với 2 chữ số của mô hình cơ bản ở bức ảnh thứ nhất và nhận diện 
sai 0 so với 4 chữ số ở bức ảnh thứ 2.

Chúng ta biết rằng Neural Structured Learning được khái quát hóa bằng hai phần chính là Neural Graph Learning và
Adversarial Learning tuy nhiên ở chương 3 chúng ta mới chỉ sử dụng Adversarial Learning để thử nghiệm, quan sát và đánh giá nó.
Qua thực nghiệm ta thấy chỉ cần thêm một nhiễu rất nhỏ mà mắt thường không thể phát hiện được, ngay lập tức đã đánh lừa được mô hình cơ bản phán đoán sai.
Điều này mang lại nhiều nguy cơ cho các hệ thống sử dụng mạng neural để phát hiện, phân loại hình ảnh như camera an ninh hay xe tự lái vì chỉ cần các bức ảnh mà camera
thu thập được không đủ tốt hoặc có nhiễu thì sẽ gây ra sai lầm cho các phán đoán cho hệ thống. Đối với các ứng dụng như nhận dạng hay phân loại đồ vật thì các sai sót xảy ra ít gây nghiêm trọng hơn, nhưng những ứng dụng như xe tự
lái, thu thập và phân tích hình ảnh theo thời gian thực, một phán đoán sai như không phát hiện người đi đường thì có thể gây ra hậu quả lớn. 
Vì vậy việc tìm ra các cách để cải thiện độ chính xác của mô hình trên các tập dữ liệu kém tương tự như Adversarial Learning là rất cần thiết.


\url{https://github.com/Danhnt0/Latex_Tieu_Luan}